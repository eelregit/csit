\documentclass[a4paper,11pt]{article}
\usepackage{jcappub}
\usepackage{aas_macros}
\usepackage{bm}
\usepackage[usenames,dvipsnames,svgnames,x11names]{xcolor}

\let\L\relax
\DeclareMathOperator{\L}{\mathcal{L}}
\DeclareMathOperator{\order}{\mathcal{O}}
\DeclareMathOperator{\deltaD}{\delta^\mathrm{D}}
\DeclareMathOperator{\erf}{erf}
\DeclareMathOperator{\erfc}{erfc}

\renewcommand{\d}{\mathrm{d}}
\newcommand{\vr}{{\bm r}}
\newcommand{\vx}{{\bm x}}
\newcommand{\vq}{{\bm q}}
\newcommand{\vk}{{\bm k}}
\newcommand{\vp}{{\bm p}}
\newcommand{\vP}{{\bm P}}
\newcommand{\vDelta}{{\bm\Delta}}
\newcommand{\rhobarm}{{\bar\rho_\mathrm{m}}}
\newcommand{\Euler}{\mathrm{E}}
\newcommand{\Lagrange}{\mathrm{L}}
\newcommand{\Tree}{\mathrm{T}}
\newcommand{\PM}{\mathrm{PM}}
\newcommand{\xs}{x_\mathrm{s}}
\newcommand{\lin}{\mathrm{lin}}
\newcommand{\eff}{\mathrm{eff}}
\newcommand{\tot}{\mathrm{tot}}
\newcommand{\sub}{\mathrm{sub}}
\newcommand{\Mpc}{\mathrm{Mpc}}
\newcommand{\Gpc}{\mathrm{Gpc}}
\newcommand{\Mpch}{{\,\Mpc/h}}
\newcommand{\hMpc}{{\,h/\Mpc}}

\newcommand{\HL}[1]{\textcolor{red}{#1}} % highlight
\newcommand{\YL}[1]{\textcolor{Bittersweet}{(YL: #1)}}
\newcommand{\KA}[1]{\textcolor{Blue}{(KA: #1)}}



\title{Cosmological simulation in tides}

%\author[a,b]{Yin Li}
%\author[]{++}
%
%\affiliation[a]{Kavli Institute for the Physics and Mathematics of the Universe
%(WPI), UTIAS, The University of Tokyo, Chiba 277--8583, Japan}
%\affiliation[b]{Berkeley Center for Cosmological Physics, Department of
%Physics, \& Lawrence Berkeley National Laboratory, University of California,
%Berkeley, CA 94720, USA}
%\emailAdd{yin.li@ipmu.jp}
%\emailAdd{yin.li@berkeley.edu}

\abstract{or other anisotropic environments.
The well-developed separate universe technique enables accurate
calibration of the response of any observable to a long-wavelength isotropic
density fluctuation.
However the large-scale environment also hosts tidal modes that perturb
all observables anisotropically.
We develop a method to include both effects in cosmological simulations.
At the level of Newtonian gravity, we redefine the simulation background to
absorb the tidal modulation on the local expansion rate, by introducing 3
different scale factors along the principal axes of a simulation box.
As an application, we measure the response of the power spectrum.
\HL{controlled experiment vs observation, causation vs correlation}
}

\begin{document}
\maketitle

\section{Introduction}
\label{sec:intro}

% observable naturally reside in RS

\section{Methodology}
\label{sec:methodology}

\subsection{Peculiarity-background split}
\label{sub:split}

\HL{Put in ZA as the simplest case, more to come in the appendix.}

In Newtonian cosmology, the effect of infinitely long-wavelength density and
tidal mode on a dark matter particle can be absorbed by a coordinate
transformation
\begin{equation}
    r_i = a_{ij} x_j,
    \label{eq:pos}
\end{equation}
where $r_i$ is the physical coordinates of the dark matter particle, and
$a_{ij}$ is a symmetric matrix that absorbs the large-scale strain so that the
large-scale displacement is isotropic in the $x_i$ coordinate.
We normalize $a_{ij}$ to the scale factor of global expansion $a\,\delta_{ij}$
in the absence of any long mode, when $x_i$ reduces to the usual comoving
coordinates.
From \eqref{pos} we can immediately separate the physical velocity $u_i$ into
the expansion of a local background and a peculiar component.
\begin{equation}
    u_i \equiv \dot r_i = H_{ij} r_j + v_i,
    \label{eq:vel}
\end{equation}
where the overdot denotes time derivative, $H_{ij} \equiv \dot a_{ik}
[a^{-1}]_{kj}$ describes a local anisotropic Hubble expansion, and $v_i\equiv
a_{ij}\dot x_j$ is the peculiar velocity.

The dark matter particle follows the Newtonian equation of motion
\begin{equation}
    \dot u_i = - \frac{\partial}{\partial r_i} (\Phi + \phi).
    \label{eq:acc}
\end{equation}
Plugging \eqref{vel} into \eqref{acc}, the acceleration also splits into
a local background expansion and a peculiar piece,
which are respectively driven by an effective background potential $\Phi$ and
a peculiar potential $\phi$
\begin{align}
    \ddot a_{ik} [a^{-1}]_{kj} r_j &= - \partial\Phi / \partial r_i,
    \label{eq:Friedmann} \\
    H_{ij} v_j + \dot v_i &= - \partial\phi / \partial r_i.
    \label{eq:pecvel}
\end{align}
One can identify \eqref{eq:Friedmann} with a modified Friedmann equation.
The background potential $\Phi$ absorb the large-scale stress due to the long
modes, so that the peculiar potential is sourced only by the local structures
\begin{equation}
    \nabla_\vr^2 \phi = 4\pi G \rhobarm (1 + \Delta_0) \delta,
    \label{eq:poisson_local}
\end{equation}
where $\rhobarm$ is the mean density of matter, $\Delta_0$ is the
local mean overdensity relative to $\rhobarm$
\begin{equation}
    1 + \Delta_0 \equiv \frac{a^3}{\det a_{ij}},
    \label{eq:D0=det}
\end{equation}
and $\delta$ denotes the overdensity with respect to the local background
density $\rhobarm (1 + \Delta_0)$.

Without loss of generality, we can simplify the equations (and the numerical
implementation) by performing a rotation to align the principal axes of the
long tide with the Cartesian axes (of the simulation box), so that $a_{ij} =
a_i \delta_{ij}$ and $H_{ij} = H_i \delta_{ij} = \dot a_i \delta_{ij} / a_i$
with their off-diagonal degrees of freedom eliminated.
Let's define $\Delta_i$ as the fractional perturbation to $a_i$
\begin{equation}
    1 + \Delta_i \equiv \frac{a_i} a, \quad i = 1, 2, 3.
\end{equation}
Therefore
\begin{equation}
    H_i = H + \frac{\dot{\Delta}_i}{1+\Delta_i}
    \simeq H +\dot{\Delta}_i,
    \label{eq:Hi}
\end{equation}
where $H$ is the global expansion rate.
The approximation holds at sufficiently high redshift when $\Delta_i \ll 1$
even if it becomes nonlinear later.

In order to determine the evolution of $\Delta_i$ in the presence of the large-scale tides $\tau_{ij}={\rm diag}(\tau_1,\tau_2,\tau_3)$,
we consider the effective background potential as
\begin{align}
    \Phi = \frac{2}{3}\pi G\bar{\rho}_{\rm m}(1+\Delta_0) r_i^2 - \frac{\Lambda}{6}r_i^2
            +2 \pi G\bar{\rho}_{\rm m}\tau_i r_i^2,
            \label{eq:ani_phi_background}
\end{align}
where the second term corresponds to the potential sourced by the long-wavelength tidal mode.
Note that we include the mean density modulation, $(1+\Delta_0)$,
which is determined through Eq.~\eqref{eq:D0=det}.
One can easily verify substituting the first two term without $(1+\Delta_0)$
into Eq.~(\ref{eq:Friedmann}) gives rise to the usual Friedmann equation.
Plugging Eq.~\eqref{eq:ani_phi_background} into Eq.~\eqref{eq:Friedmann}
and using the usual Friedmann equation,
we derive
\begin{align}
    \ddot \Delta_i + 2H\dot \Delta_i
    = -4\pi G \bar{\rho}_m\left(\tau_i + \frac{1}{3}\Delta_0\right) (1+\Delta_i).
    \label{eq:Deltai_eq}
\end{align}
This is analogous equation for the modified scale factors to Eq.~(10) in Ref.~\cite{Wagner_etal:2014},
while they consider the isotropic modulation.
Indeed, setting $\tau_i=0$ reproduces Eq.~(10) in Ref.~\cite{Wagner_etal:2014}.
Linerizing Eq.~\eqref{eq:Deltai_eq} yields
\begin{align}
 \ddot \Delta_i + 2H\dot \Delta_i
 = -\frac{3}{2}H^2\Omega_{\rm m}\tau_i,
\end{align}
where we used the Friedmann equation and $\Delta_0 = {\cal O}(\Delta_i^2)$.
Given that $\tau_i = (D(t)/D_0)\tau_{i0}$,
this equation is exactly the same differential equation for the linear growth factor
except for the sign of the right-hand-side.
Thus, at linear order we get
\begin{align}
\Delta_i(t) = -\tau_i(t) \propto D(t).
\end{align}
\KA{
The second order solution should be $\Delta^{(2)}_0 = \frac{2}{7}\sum_i\tau_i^2$ and I checked it numerically, but I cannot derive it by now.
}
\YL{
At second order
\begin{equation}
    \ddot\Delta_i^{(2)} + 2H \dot\Delta_i^{(2)}
    = 4\pi G \bar\rho_m \Bigl( {\Delta_i^{(1)}}^2 - \frac13 \Delta_0^{(2)''} \Bigr)
\end{equation}
Use
\begin{equation}
    \Delta_0^{(2)} = - \sum_i \Delta_i^{(2)}
    + \sum_{i<j} \Delta_i^{(1)} \Delta_j^{(1)} + \sum_i {\Delta_i^{(1)}}^2
    = - \sum_i \Delta_i^{(2)} + \frac12 \sum_i {\Delta_i^{(1)}}^2
\end{equation}
Sum over $i$ of the first equation and then plug in the second one, we can show that
at matter-dominated era
\begin{equation}
    \sum_i \Delta_i^{(2)} = \frac3{14} \sum_i {\Delta_i^{(1)}}^2
\end{equation}
therefore
\begin{equation}
    \Delta_0^{(2)} = \frac27 \sum_i \Delta_i^{(2)}
\end{equation}
}
%For later convinience, we compute the second order solution in the Einstein de-Sitter universe.
%The second order equation is given by
%\begin{align}
% \ddot \Delta^{(2)}_i + 2H\dot \Delta^{(2)}_i
% = \frac{3}{2}H^2\Omega_{\rm m}\left(\Delta^{(1)}_i\right)^2,
% \label{eq:Deltai_2nd_eq}
%\end{align}
%where we set $\tau_i = -\Delta_i^{(1)} $, which is valid at this order.
%The solution for Eq.\eqref{eq:Deltai_2nd_eq} is found to be
%\begin{align}
%\Delta_i^{(2)} = \frac{1}{2}\left[ \Delta_i^{(1)} \right]^2.
%\end{align}
For anisotropic simulations we compute the anisotropic scale factor
by solving Eq.\eqref{eq:Deltai_eq} numerically.

Defining the conjugate momenta of $x_{i}$ as $P_{i}\equiv a_i^2 m \dot x_{i}$,
the equation of motion for the peculiar part takes a simple form of
\begin{align}
    \dot P_{i} = - \frac{\partial \phi}{\partial x^i},
    \label{eq:mod_EoM}
\end{align}
for the unit mass.



\subsection{Lagrangian perturbation theory}
\label{sub:lpt}

In the presence of long modes, we solve the anisotropic perturbations to the
Lagrangian displacement with the second order Lagrangian perturbation theory,
which we will use later to generate the initial conditions.

Let us start with the Zel'dovich approximation (or linear Lagrangian
perturbation theory)~\cite{Zeldovich:1969}.
The displacement field $\Psi({\vq})$ is a mapping from a particle's Lagrangian
position $q_i$ to its Eulerian position $x_i$:
\begin{equation}
    x_i = q_i + \Psi_i({\vq}),
    \label{eq:disp}
\end{equation}
whose determinant is related to overdensity by
\begin{equation}
    \delta = \Bigl| \frac{\partial{\vx}}{\partial{\vq}} \Bigr|^{-1} - 1
    \simeq - \Psi_{i,i}^{(1)},
    \label{eq:J}
\end{equation}
where $\Psi_{i,j} \equiv \partial\Psi_i / \partial q_j$, and we have only kept
the leading order term in the second equality.
Substitute Eq.~\eqref{eq:disp} into Eq.~\eqref{eq:pecvel} to derive
\begin{equation}
    \ddot \Psi_i^{(1)} + 2 H_i \dot \Psi_i^{(1)} = - \frac1{a_i}
    \frac{\partial\Phi}{\partial r_i}.
\end{equation}
Take gradient of $x_i$ on both sides and then sum over $i$
\begin{equation}
    \ddot \Psi_{i,i}^{(1)} + 2 H_i \dot \Psi_{i,i}^{(1)} \simeq - \nabla_{\vr}^2 \Phi
    \simeq \frac{3}{2}H^2 \Omega_{\rm m}(a)  (1 + \Delta_0)\Psi_{i,i}^{(1)},
 %   \frac{3\Omega_\mathrm{m} H_0^2}{2 a^3} (1 + \Delta_0),
    \label{eq:disp_evo}
\end{equation}
in which we have used Eqs.~\eqref{eq:poisson_local}, \eqref{eq:J}, and the fact that at leading
order $\partial\Psi_i^{(1)} / \partial x_i \simeq \Psi_{i,i}^{(1)}$
and $\Omega_{\rm m}(a) =  \Omega_{\rm m}H_0^2/a^3H^2$.
In the linear order the vorticity in $\Psi_{i,j}^{(1)}$ decays, so the displacement
is a potential flow $\Psi_i^{(1)} = - \partial\psi^{(1)}/ \partial q_i \equiv -\psi_{,i}^{(1)}$
with the displacement potential $\psi^{(1)}$ sourced by the overdensity in Lagrangian
space:
\begin{equation}
    \nabla_{\vq}^2 \psi^{(1)} \simeq \delta^{(1)}.
\end{equation}
Rewriting Eq.~\eqref{eq:disp_evo} with $\psi$ in Fourier space yields
\begin{equation}
   \ddot \psi^{(1)} + 2 H_i \hat p_i^2   \dot\psi^{(1)}
    - \frac32 \Omega_\mathrm{m}(a) (1 + \Delta_0) \psi^{(1)} = 0,
    \label{pot_evo}
\end{equation}
where $\hat p_i$ denotes the unit vector of the wavevector
in the Lagrangian space.
%, and $\Omega_\mathrm{m}(a) \equiv \Omega_\mathrm{m}
%H_0^2 / a^3 H^2$.
The above equation describes the linear growth of the displacement field, on
which the effect of the long modes manifests in the quadrupolarly
direction-dependent Hubble drag and the coefficients that depend on the growth
history of the long modes $\Delta_i$(t).

Apparently without the long modes the evolution described by Eq.~\eqref{pot_evo} reduces to
that of the usual linear growth function $D(t)$,
\begin{align}
    \ddot{D}^{(1)} + 2H\dot{D}^{(1)} -\frac{3}{2}\Omega_{\rm m}(a)D^{(1)}{(t)}=0.
    \label{eq:linear_growth_function}
\end{align}
In the presence of the long modes, the growth function $D^{(1)}_W(a, \hat{p})$
receives corrections of order ${\cal O}(\Delta_i)$.
The subscript $W$ here
denotes locally averaged quantities
defined in a \emph{window} with non-vanishing long modes.
Let $\epsilon^{(1)}(a, \hat{\vp}) \equiv D^{(1)}_W - D^{(1)}$. It follows
\begin{equation}
    \ddot \epsilon^{(1)} +2H \dot \epsilon^{(1)}
    - \frac32 \Omega_{\rm m}(a) \epsilon^{(1)}
    \simeq
    - 2\dot D^{(1)} \hat p_i^2 \dot \Delta_i + \frac32 \Omega_{\rm m}(a) D^{(1)} \Delta_0.
\end{equation}
To solve this equation we can first decompose it as
\begin{equation}
    \epsilon^{(1)}(a, \hat{\vp}) = \hat p_i^2 \epsilon^{(1)}_i(a),
    \label{e}
\end{equation}
with each component $\epsilon^{(1)}_i(a)$ solves
\begin{equation}
    \ddot \epsilon_i^{(1)} + 2H \dot \epsilon_i^{(1)}
    - \frac32 \Omega_{\rm m}(a)\epsilon_i^{(1)}
    = - 2 \dot D^{(1)} \dot \Delta_i + \frac32 \Omega_{\rm m}(a) D^{(1)} \Delta_0.
    \label{eq:ei}
\end{equation}
For the simplest scenarios we assume that the universe has been matter
dominated ($H = 2/3t$ and $\Omega_{\rm m}(a) = 1$) for long enough and
that the long modes are well sub-horizon ($\Delta_0, \Delta_i \propto D^{(1)}$), so
that at the Zel'dovich approximation we can set up the initial conditions of the linear growth equations
Eqs.~\eqref{eq:linear_growth_function} and \eqref{eq:ei} as
\begin{align}
    \epsilon^{(1)}_i &=
    -\frac47 \Delta_i(a_\mathrm{ini}) D^{(1)}(a_\mathrm{ini}) + \frac37 \Delta_0(a_\mathrm{ini}) D^{(1)}(a_\mathrm{ini}),
    \nonumber\\
    \frac{\d \epsilon^{(1)}_i}{\d t} &= 2 H(a_{\rm ini})f_1(a_{\rm ini})\epsilon^{(1)}_i,
    \label{eq:ICs_ZA}
\end{align}
with $f_1\equiv \d \ln D^{(1)}/\d \ln a \simeq $
at $a_\mathrm{ini}$ deep in the matter-dominated era.

eq.~\eqref{eq:ICs_ZA} can be directly compared to the results derived with standard
perturbation theory in an Einstein de-Sitter universe.
For isotropic perturbation, $\Delta_i = - \Delta_0 / 3$, so $D^{(1)}_W = D^{(1)} (1 + 13
\Delta_0 / 21)$.

At second order, we have to come back to the master equation for the Lagrangian perturbation theory
in the anisotropic background, which is given by
\begin{align}
    \Bigl| \frac{\partial\vx}{\partial\vq} \Bigr|^{-1}
    \left[ \delta_{ij} + \Psi_{i,j} \right]^{-1}
    \left[ \ddot{\Psi}_{i,j} + 2H_i\dot{\Psi}_{i,j} \right]
    =\frac32 H^2\Omega_{\rm m}(a)(1+\Delta_0)\delta.
\end{align}
Keeping up to the second order terms, we find
\begin{align}
&\Bigl| \frac{\partial\vx}{\partial\vq} \Bigr|^{-1}  \simeq
    1 + \Psi_{i,i}^{(1)} + \Psi_{i,i}^{(2)} + \frac12 \left[\left(\Psi_{i,i}^{(1)} \right)^2 - \Psi_{i,j}^{(1)}\Psi_{j,i}^{(1)} \right]
    \\
&\left[ \delta_{ij} + \Psi_{i,j} \right]^{-1} \simeq \delta_{ij} - \Psi_{i,j}^{(1)},
\end{align}
leading to the second-order equation
\begin{align}
\ddot \Psi^{(2)}_{i,i} &+ 2H_i\dot \Psi^{(2)}_{i,i} -\frac32 H^2\Omega_{\rm m}(a)(1+\Delta_0)\Psi_{i,i}^{(2)}
    \nonumber\\
    &= -\Psi_{i,i}^{(1)}\left[\ddot \Psi^{(1)}_{j,j} + 2H_i\dot \Psi_{j,j}^{(1)} \right]
    + \Psi_{i,j}^{(1)}\left[\ddot \Psi^{(1)}_{i,j} + 2H_i\dot \Psi_{i,j}^{(1)} \right]
    \nonumber\\
    &\hspace{1cm}+ \frac32 H^2 \Omega_{\rm m}(a) (1+\Delta_0) 
    \left[\frac12\left(\Psi_{i,i}^{(1)}\right)^2 - \frac12\Psi_{i,j}^{(1)}\Psi_{j,i}^{(1)}  \right].
\end{align}
Similar to the linear order, we introduce the second order displacement potential
through $\Psi_i^{(2)} = \partial\psi_W^{(2)}/\partial q_i \equiv \psi^{(2)}_{W,i}$.
In the absence of the long modes, the equation for $\psi^{(2)}$ reduces
\begin{align}
\ddot \psi^{(2)}_{,ii} + 2H\dot \psi^{(2)}_{,ii}-\frac32 H^2 \Omega_{\rm m}(a)\psi^{(2)}_{,ii}
 = -\frac32 H^2 \Omega_{\rm m}\left[\frac12 \left( \psi^{(1)}_{,ii} \right)^2 -\frac12 \psi^{(1)}_{,ij}\psi^{(1)}_{,ji} \right],
\end{align}
where we used the linear equation \eqref{pot_evo} without long modes.
In this usual case, we denote the time-dependent part of $\psi^{(2)}$ as $D^{(2)}(t)$, which obeys
\begin{align}
\ddot D^{(2)} + 2H \dot D^{(2)} - \frac32 H^2 \Omega_{\rm m}(a) D^{(2)} = -\frac32 H^2 \Omega_{\rm m}\left( D^{(1)} \right)^2.
\end{align}
In the matter dominated era, we have $D^{(2)}= -\frac37 \left[D^{(1)}\right]^2$.
The correction induced by the long modes, which is expressed by $\epsilon^{(2)}(t, \vq)\equiv \psi_W^{(2)}(t, \vq) - \psi^{(2)}(t, \vq)$,
follows
\begin{align}
\ddot \epsilon^{(2)}_{,ii} &+ 2H \dot \epsilon^{(2)}_{,ii} - \frac32 H^2 \Omega_{\rm m}(a) \epsilon_{,ii}^{(2)}=
\nonumber\\
&-2\dot\psi^{(2)}_{,ii} \dot\Delta_i + 2\dot\Delta_i\dot\psi_{,ij}^{(1)}\psi_{,ji}^{(1)}
-\frac32 H^2 \Omega_{\rm m}(a) \epsilon^{(1)}_{,jj}\psi^{(1)}_{,ii} + \psi^{(1)}_{,ij} \left[ \ddot\epsilon_{,ij}^{(1)} + 2H\dot\epsilon_{,ij}^{(1)} \right]
\nonumber\\
&+\frac32 H^2 \Omega_{\rm m}(a)\Delta_0\left[ \psi_{,ii}^{(2)} + \frac12\left( \psi^{(1)}_{,ii}\right)^2- \frac12\psi^{(1)}_{,ij}\psi^{(1)}_{,ji} \right],
\label{eq:2nd_epsilon}
\end{align}
where $\epsilon^{(1)}(t,\vq) = \int_\vp \epsilon^{(1)}(a, \vp)e^{i\vp\cdot\vq}$
and we neglect $\mathcal{O}(\Delta_i^2)$ terms.
Notice that $\epsilon^{(1)}$ is $\mathcal{O}(\Delta_i)$.
For the matter dominated era, in which we have $H=2/3t$ and $\Omega_{\rm m}(a)=1$,
%$\Delta_0, \Delta_i, \psi^{(1)} \propto D^{(1)}$ and $\psi^{(2)},\epsilon^{(1)}\propto D^{(2)}$
the solution for Eq.\eqref{eq:2nd_epsilon} is given by
\begin{align}
\epsilon^{(2)}_{,ii}(t, \vq)
=&\frac14 \left[-\frac{16}{9}\psi^{(2)}_{,ii}(t,\vq) + \frac89\psi^{(1)}_{,ij}(t,\vq)\psi^{(1)}_{,ji}(t,\vq) \right]\Delta_i
\nonumber\\
&+\frac16 \left[\psi_{,ii}^{(2)} + \frac12\left( \psi^{(1)}_{,ii}\right)^2- \frac12\psi^{(1)}_{,ij}\psi^{(1)}_{,ji}   \right]\Delta_0
\nonumber\\
&+\frac14\left[-\frac23\psi^{(1)}_{,ii}(t,\vq)\epsilon_{,jj}^{(1)}(t,\vq) + \frac{20}{9}\psi^{(1)}_{,ij}(t,\vq)\epsilon_{,ij}^{(1)}(t,\vq)  \right].
\end{align}
Although the modified second order growth factor, $D^{(2)}_W$, due to the long modes can be identified as $D^{(2)}_W (t, \vp) = D^{(2)}(1 + \Delta_0/6 - 4\hat{p}_i^2\Delta_i/9 )$, the local gravitational tides cannot be neglected at second order.


\subsection{Power spectrum responses}
\label{sub:resp}


Alternatively one can parametrize the remaining 3 degrees of freedom of the
background strain by its isotropic dilation and shears \cite{BondMyers96I},
with the former being approximately the negative mean overdensity
\begin{align}
    -\Delta_0 &\simeq \Delta_1 + \Delta_2 + \Delta_3, \nonumber\\
    \Delta_\mathrm{e} &\equiv \frac{\Delta_1 - \Delta_2}2, \nonumber\\
    \Delta_\mathrm{p} &\equiv \Delta_3 - \frac{\Delta_1 + \Delta_2}2.
\end{align}
The subscripts of the two shear modes stand for ellipticity and
prolaticity\footnote{We do not impose ordering on $\vDelta$ to compress the
parameter space (cf.\ \cite{BondMyers96I}).}.


We recycle the symbol $\vp$ here to Eulerian\KA{Lagrangian?} space to denote the wavevector in
local comoving coordinates, as it coincides with the Lagrangian\KA{Eulerian?}  space
wavevector in the large scale limit.

We consider the multipole expansions
\begin{align}
    - \frac{\partial\ln P}{\partial\Delta_i} \bigg|_\vk
    &\equiv R^\Euler_0(k) + \L_2(\hat k_i) R^\Euler_2(k), \nonumber\\
    - \frac{\partial\ln P_W}{\partial\Delta_i} \bigg|_\vp
    &\equiv R^\Lagrange_0(p) + \L_2(\hat p_i) R^\Lagrange_2(p),
\end{align}
where holding the other two $\Delta_j$'s ($j \neq i$) fixed.
$\L_2$ is the second order Legendre polynomial.
Similar to the decomposition of halo biases, we name the Eulerian and
Lagrangian responses, and distinguish them with superscripts.

We need to be careful because the scales shift.
Let $k_i$ be the Eulerian space wavevector, and $p_i$ be the wavevector of the
initial conditions in the Lagrangian space.
The two are related by the conversion with the global and local scale factors
\begin{equation}
    p_i = \frac{a_i}a k_i = (1 + \Delta_i) k_i,
\end{equation}
so that $\vk = \vp$ when $\vDelta = \bm0$.
Consider the change of reference density and the conservation of variance when
transforming between Fourier-space volume elements
\begin{equation}
    P \,\d^3\vk = (1 + \Delta_0)^2 P_W \,\d^3\vp,
\end{equation}
so
\begin{equation}
    P = (1 + \Delta_0) P_W
\end{equation}

$P_W(\vp, \vDelta)$

\begin{align}
    - \frac{\partial\ln P}{\partial\Delta_i} \bigg|_\vk
    = 1 - \frac{\partial\ln P_W}{\partial\Delta_i} \bigg|_\vk
    &= 1 - \frac{\partial\ln P_W}{\partial\Delta_i} \bigg|_\vp
    - \frac{\partial\ln P_W}{\partial p_j}
        \frac{\partial p_j}{\partial \Delta_i} \bigg|_{\vk, \vDelta=\bm0}
    \nonumber\\
    &= 1 - \frac{\partial\ln P_W}{\partial\Delta_i} \bigg|_\vp
    - \hat k_i^2 \frac{\d\ln P}{\d\ln k}, \nonumber\\
    &= 1 - \frac{\partial\ln P_W}{\partial\Delta_i} \bigg|_\vp
    - \frac13 \frac{\d\ln P}{\d\ln k}
    - \L_2(\hat k_i) \frac23 \frac{\d\ln P}{\d\ln k},
\end{align}
where $\hat k_i = k_i / k$.
Therefore
\begin{align}
    R^\Euler_0 &= 1 + R^\Lagrange_0 - \frac13 \frac{\d\ln P}{\d\ln k},
    \nonumber\\
    R^\Euler_2 &= R^\Lagrange_2 - \frac23 \frac{\d\ln P}{\d\ln k}.
\end{align}

For either response
\begin{align}
    R_0 &= R_0, \nonumber\\
    R_\mathrm{e}
    &= - \bigl(\L_2(\hat p_1) - \L_2(\hat p_2)\bigr) R_2, \nonumber\\
    R_\mathrm{p} &= - \L_2(\hat p_3) R_2.
\end{align}
In the linear regime $P_W \propto D_W^2$ and
\begin{equation}
    D_W = D \Bigl( 1 + \frac{13}{21} \Delta_0
    - \frac8{21} \L_2(\hat p_i) \Delta_i \Bigr),
    \label{DW}
\end{equation}
so that the tree-level responses are
\begin{align}
    R^\Lagrange_0 = \frac{26}{21}, \qquad
    & R^\Euler_0 = \frac{47}{21} - \frac13 \frac{\d\ln P}{\d\ln k};
    \nonumber\\
    R^\Lagrange_2 = \frac{16}{21}, \qquad
    & R^\Euler_2 = \frac{16}{21} - \frac23 \frac{\d\ln P}{\d\ln k}.
\end{align}
These results are consistent with the standard perturbation theory calculation
in the previous works for example \cite{LiSchmittfullSeljak17}\footnote{In
\cite{LiSchmittfullSeljak17} the dilation piece is written as the slope of the
dimensionless power spectrum, thus the constant terms become $47/21 \to 68/21$
and $16/21 \to 58/21$ there.}.

\HL{Should there be $R_\ell$ with $\ell>2$? Taylor expansion suggests
no. But deep in nonlinear structures?}



\section{Numerical implementation}
\label{sec:num}
In this section, we describe how to modify
the comsmological $N$-body simulation code \texttt{L-Gadget2}~\cite{Springel:2005},
to perform the cosmological simulation in an anisotropic expanding universe.


\subsection{Initial conditions}
\label{sub:ics}

\KA{Almost written in Sec.2.2?}


\subsection{Forces}
\label{sub:treepm}

The TreePM method splits the gravity into the long-range and short-range pieces
\begin{equation}
    \phi = \phi^\PM + \phi^\Tree,
\end{equation}
which are handled by the PM and tree algorithms respectively \cite{Bagla02,
BaglaRay03}
\begin{align}
    \phi^\PM(\vk) &= - 4\pi G \rhobarm a^2
        \frac{\delta(\vk)}{k^2} e^{- k^2 \xs^2} \\
    \phi^\Tree(\vx) &= - \frac{G m}a \sum_n \frac1{|\vx-\vx_n|}
    \erfc \Bigl( \frac{|\vx-\vx_n|}{2 \xs} \Bigr)
\end{align}
where $\xs$ is the comoving scale of force splitting,
$\vx_n$ denotes the position of the $n$-th particle,
and
\begin{equation}
    \rhobarm \bigl( 1 + \delta(\vx) \bigr)
    = \frac{m}{a^3} \sum_n \deltaD(\vx-\vx_n).
\end{equation}
Since the Fourier transform of $e^{- k^2 x_s^2} / k^2$ is ${\rm erf}(x / x_s) / 4\pi x$,
one can verify the force splitting with the convolution theorem.
The acceleration due to the tree force is
\begin{equation}
    - \nabla_\vx \phi^\Tree = - \frac{G m}a \sum_n
    \frac{\vx - \vx_n}{|\vx - \vx_n|^3}
    \biggl[ \erfc \Bigl( \frac{|\vx - \vx_n|}{2 \xs} \Bigr)
            + \frac{|\vx - \vx_n|}{\xs \sqrt\pi}
            \exp \Bigl( - \frac{|\vx - \vx_n|^2}{4 \xs^2} \Bigr) \biggr].
\end{equation}

Now let us consider how this conventional TreePM algorithm is modified
in an anisotropically expamding universe.

The modified Poisson equation Eq.~\eqref{eq:poisson_local},
\begin{equation}
    (1+\Delta_i)^{-2} \frac{\partial^2}{\partial x_i^2} \phi^{\rm PM}
    = 4\pi G \rhobarm a^2 (1 + \Delta_0) \delta.
\end{equation}
leads us to modify the TreePM potentials as
%Compare to the usual case, we modify the TreePM algorithm
\begin{align}
    \phi^\PM(\vp) &= - 4\pi G \rhobarm a^2 (1 + \Delta_0)
        \frac{\delta(\vp)}{k^2} e^{- k^2 \xs^2}, \\
    \phi^\Tree(\vx) &= - \frac{G m}a \sum_n \frac{a}{|\vr - \vr_n|}
    \erfc \Bigl( \frac{|\vr - \vr_n|}{2 a \xs} \Bigr).
\end{align}
Recall that $k_i = p_i / (1 + \Delta_i)$ is the global comoving wavevector
and $r_i = a_i x_i$ is the physical position.
Note that to not introduce anisotropic artificial features, we split the force
isotropically in physical scales, i.e.\ anisotropically in local comoving
scales.
Given that the PM grid is now anisotropic in physical scale, having $\xs$
bigger than the grid size helps to suppress any effect due to such anisotropic
smoothing at grid scale.
The tree acceleration now is
\begin{equation}
    - \frac{\partial\phi^\Tree}{\partial x_i} = - \frac{G m}a \sum_n
    \frac{a a_i [\vr - \vr_n]_i}{|\vr - \vr_n|^3}
    \biggl[ \erfc \Bigl( \frac{|\vr - \vr_n|}{2 a \xs} \Bigr)
            + \frac{|\vr - \vr_n|}{a \xs \sqrt\pi}
            \exp \Bigl( - \frac{|\vr - \vr_n|^2}{4 a^2 \xs^2} \Bigr) \biggr],
\end{equation}
We adopt the same spline kernel in \texttt{L-Gadget2} to soften the force
with replacement $\vx \to \vr / a$.







\subsection{Time integration}
\label{sub:integ}

In simulations \cite{QuinnKatzEtAl97, Springel05}, the kick and drift leapfrog
operators
\begin{align}
    \mathrm{Kick} &: \quad \frac{\vP}m \to \frac{\vP}m
        - \nabla_\vx \phi \int_t^{t+\Delta t} \frac{\d t}a  \\
    \mathrm{Drift} &: \quad \vx \to \vx
        + \frac{\vP}{m} \int_t^{t+\Delta t} \frac{\d t}{a^2}
\end{align}
are derived from canonical transformation.
$\vP = a^2 m \dot \vx$ is the canonical momentum of the Hamiltonian
\begin{equation}
    H = \sum_n \frac{\vP_n^2}{2 m a^2}
    + \sum_{n \neq n'} \frac{m^2 \varphi(\vx_n - \vx_{n'})}{2a},
\end{equation}
in which $\varphi$ is the potential of a unit-mass particle at $a=1$:
\begin{equation}
    \nabla_\vx^2 \varphi = 4\pi G \Bigl(\deltaD(\vx-\vx') - \frac1{L^3} \Bigr).
\end{equation}

For the anisotropic system,
%we have the modified canonical momentum $P_i\equiv a_i^2 m \dot x_i$.
the Hamiltonian leading to the modified EoM Eq.~\eqref{eq:mod_EoM} is
%\begin{equation}
%    H = \sum_n \sum_{i=1}^3 \frac{P_{n,i}^2}{2 m a_i^2}
%    + \sum_{n \neq n'} \frac{G m^2}{2 |\vr_n - \vr_{n'}|}
%\end{equation}
\begin{equation}
    H = \sum_n \sum_{i=1}^3 \frac{P_{n,i}^2}{2 m a_i^2}
    + \sum_{n \neq n'} \frac{m^2 \varphi({\bf x}_n - {\bf x}_{n'})}{2a},
\end{equation}
where $P_{n,i}\equiv a_i^2 m \dot x_{n,i}$ are the conjugate momenta of $x_{ni}$.
Thus we have
\begin{align}
    \mathrm{Kick} &: \quad \frac{\vP}m \to \frac{\vP}m
        - \nabla_{\bf x} \varphi \int_t^{t+\Delta t} \frac{d t}a,  \\
    \mathrm{Drift} &: \quad x_i \to x_i
        + \frac{P_i}{m} \int_t^{t+\Delta t} \frac{d t}{a_i^2}.
\end{align}
Note that since we have already considered effects of the anisotropic expansion 
on force (or potential) calculations in the previous subsection, we can use the same kick operators.
On the other hand, the drift operator should be modified due to the modified canonical momentum.

%We use the same the kick and drift operators, which should serve as a good approximation
%to the symplectic time integration,
%as the time dependence in the Hamiltonian has been mostly captured by the global scale
%factor $a$.




\section{Results}
\label{sec:results}


Power spectrum responses


Convergence tests: mass res, force res, force split.

\section{Halo tidal bias}
\label{sec:bias}
\subsection{recap of $b_1$ and $b_2$ as response bias in separate universe simulations}

\begin{align}
n_{\ln{M}} = \frac{N(\ln{M})}{V_W},
\end{align}
where $n_{\ln{M}}$ and $N(\ln{M})$ are the differential number density of halos per logarithemic mass interval and 
the differential number of halos per logarithemic mass interval, respectively.
$V_W$ is the comoving volume of each separate universe, which is related to the comoving volume of the  fidutial universe $V$, such that $V = (1+\Delta_0) V_W$ implied by the mass conservation.
The first order Eulerian bias is defined through
\begin{align}
b_{1}(M) =& \frac{1}{n_{\ln{M}}}
\frac{\d n_{\ln{M}}}{\d \Delta_0}
=\frac{1}{n_{\ln{M}}} \left.\frac{\partial n_{\ln{M}}}{\partial\Delta_0}\right|_{V_W}
+ \frac{1}{n_{\ln{M}}}\left.\frac{\partial n_{\ln{M}}}{\partial \Delta_0}\right|_{N}
\\
=& \frac{1}{N(\ln{M})}
\left.\frac{\partial N(\ln{M})}{\partial \Delta^{(1)}_0}\right|_{V_W}
+\frac{\partial}{\partial \Delta_0}\left(1+\Delta_0\right),
\label{eq:b1_relation}
\end{align}
where we used $\Delta_0 = \Delta_0^{(1)}$ at first order.
We identify the first term in Eq.~(\ref{eq:b1_relation}) as 
the first order Lagrangian bias $b_1^{\rm L}(M)$. 
Since the second term clearly gives 1, we have
\begin{align}
b_1(M) = b_1^{\rm L}(M) + 1.
\end{align}
At second order, we should take into account 
the nonlinear evolution of $\Delta_0$,
which can be described by the spherical collapse,
\begin{align}
\Delta_0 =& \Delta_0^{(1)} + \frac{17}{21}\left(\Delta_0^{(1)}\right)^2 + {\cal O}\left[
\left(\Delta^{(1)}_0\right)^3\right],
\\
\Delta_0^{(1)} =& \Delta_0 - \frac{17}{21}\left(\Delta_0\right)^2 
+ {\cal O}\left[ \left(\Delta_0\right)^3\right].
\end{align}
Note that this nonlinear evolution is naturally realized in the isotropic separate universe simulation tequnichs.
We derive 
\begin{align}
\frac{\d}{\d \Delta_0} 
=&
\frac{\d \Delta_0^{(1)} }{\d \Delta_0}\frac{\d }{\d \Delta_0^{(1)}}
=\left( 1 - \frac{34}{21}\Delta_0 \right)\frac{\d }{\d \Delta_0^{(1)}},
\\
\left. \frac{\d^2}{\d \Delta_0^2}\right|_{\Delta_0=0}
=& \frac{\d^2}{\left(\d \Delta_0^{(1)}\right)^2}
-\frac{34}{21}\frac{\d }{\d \Delta_0^{(1)}}.
\end{align}
The second order Eulerian bias is defined as
\begin{align}
b_2(M) = \frac{1}{n_{\ln{M}}}\frac{\d^2 n_{\ln{M}}}{\d \Delta_0^2}.
\end{align}
Then, similary to the first order case, by using
$ \d/\d\Delta_0^{(1)}=
\left.\partial/\partial\Delta_0^{(1)}\right|_{V_W} 
+\left.\partial/\partial\Delta_0^{(1)}\right|_{N}$,
we get
\begin{align}
b_2(M)
=&\frac{1}{n_{\ln{M}}}
\left[ \left.\frac{\partial^2 n_{\ln{M}} }{\left(\partial \Delta_0^{(1)}\right)^2}\right|_{V_W} 
+ 2 \left.\frac{\partial}{\partial \Delta_0^{(1)}}\right|_{V_W}
\left.\frac{\partial}{\partial \Delta_0^{(1)}}\right|_{N}n_{\ln{M}}
+ \left.\frac{\partial^2 n_{\ln{M}}  }{\left(\partial \Delta_0^{(1)}\right)^2}\right|_{N} \right]
-\frac{34}{21}
\frac{1}{n_{\ln{M}}}
\frac{\d n_{\ln{M}}}{\d \Delta_0^{(1)} }
\nonumber\\
=&\frac{1}{n_{\ln{M}}}\left.\frac{\partial^2 n_{\ln{M}} }{\left(\partial \Delta_0^{(1)}\right)^2}\right|_{V_W} 
+2b_1^{\rm L}(M) + \frac{34}{21} 
- \frac{34}{21}\left( b_1^{\rm L}(M) + 1 \right)
\nonumber\\
=&b_2^{\rm L}(M) + \frac{8}{21}b_1^{\rm L}(M),
\end{align}
where we identify the first term in the second line as the second order Lagrangian bias.
In fact, this term can be directly measured from separate universe simulations in the fixed comoving volume.

\begin{align}
n_{\ln{M}}(\Delta_{0}) \equiv
\frac{N_{\ln{M}}(\Delta_{0})}{V(\Delta_{0})}
\end{align}

\begin{align}
b^{\rm E}_{1}
= \frac{1}{n_{\ln{M}}(\Delta_0=0)}
\frac{n_{\ln{M}}(\Delta_0=\epsilon)
-
n_{\ln{M}}(\Delta_0=-\epsilon)
}{2\Delta_0}
\end{align}


\begin{align}
b^{\rm L}_{1}
= \frac{1}{N_{\ln{M}}(\Delta_0=0)}
\frac{N_{\ln{M}}(\Delta_0=\epsilon)
-
N_{\ln{M}}(\Delta_0=-\epsilon)
}{2\Delta_0}
\end{align}


\begin{align}
b^{\rm E}_{2}
= \frac{1}{n_{\ln{M}}(\Delta_0=0)}
\frac{n_{\ln{M}}(\Delta_0=\epsilon)
+
n_{\ln{M}}(\Delta_0=-\epsilon)
-2n_{\ln{M}}(\Delta_0=0)}{\Delta^2_0}
\end{align}

\begin{align}
b^{\rm L}_{2}
= \frac{1}{N_{\ln{M}}(\Delta_0=0)}
\frac{N_{\ln{M}}(\Delta_0=\epsilon)
+
N_{\ln{M}}(\Delta_0=-\epsilon)
-2N_{\ln{M}}(\Delta_0=0)}{\Delta^2_0}
\end{align}


\subsection{tidal bias : $b_{s^2}$}

The tidal bias can also be interpreted as the response of the halo number density
to the long-wavelength tidal field.

\begin{align}
n_{\ln{M}}(\Delta_{\rm p}) \equiv
\frac{N_{\ln{M}}(\Delta_{\rm p})}{V(\Delta_{\rm p})}
\end{align}

\begin{align}
b^{\rm E}_{s^2}
= \frac{1}{n_{\ln{M}}(\Delta_{\rm p}=0)}
\frac{n_{\ln{M}}(\Delta_{\rm p}=+2\epsilon)
+
n_{\ln{M}}(\Delta_{\rm p}=-2\epsilon)
-2n_{\ln{M}}(\Delta_{\rm p}=0)}{3\Delta^2_{\rm p}/8}
\end{align}


\begin{align}
b^{\rm L}_{s^2}
= \frac{1}{N_{\ln{M}}(\Delta_{\rm p}=0)}
\frac{N_{\ln{M}}(\Delta_{\rm p}=+2\epsilon)
+
N_{\ln{M}}(\Delta_{\rm p}=-2\epsilon)
-2N_{\ln{M}}(\Delta_{\rm p}=0)}{3\Delta^2_{\rm p}/8}
\end{align}


Since in our anisotropic separate universe simulations we have homogeneous tidal field whose configuration is 
\begin{align}
s_{ij} = \tau_{ij} = \mathrm{diag}(-\epsilon/2,-\epsilon/2,\epsilon),
\end{align}
$(s_{ij})^2$ is given by $3\epsilon^2/2$.
Note that the minus configration $\tau_{ij} = \mathrm{diag}(\epsilon/2,\epsilon/2,-\epsilon)$ also gives $(s_{ij})^2 = 3\epsilon^2/2$.
Thus 
\begin{align}
b_{s^2}^{\rm L}(M) = \left.\frac{\partial^2 \ln{ n_{\ln{M} } } }{\partial s_{ij}^2}\right|_{V_{\rm com}}=
\frac{\ln{ n_{\ln{M} }(\tau_{ij}) } -\ln{ n_{\ln{M} }(\tau_{ij}=0) }  }{\frac{3}{4}\epsilon^2}
\end{align}

\section{Discussion}
\label{sec:discuss}


\acknowledgments{I thank Yu Feng, Takahiro Nishimichi for helpful discussions.
I acknowledge support from Fellowships at the Kavli IPMU established by World
Premier International Research Center Initiative (WPI) of the MEXT Japan, and
at the Berkeley Center for Cosmological Physics.
This research used resources of the National Energy Research Scientific
Computing Center, a DOE Office of Science User Facility supported by the Office
of Science of the U.S.~Department of Energy under Contract
No.~DE-AC02-05CH11231.
}


\appendix



\section{Models}
\label{sec:models}


\subsection{Ellipsoidal collapse}
\label{sub:ellip}

\subsection{Local tide approximation}
\label{sub:lta}




%\bibliographystyle{JHEP}
%\bibliography{cosmo,cosmo_preprints}
\end{document}
